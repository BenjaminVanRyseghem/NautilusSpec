\documentclass[12pt]{article}
\begin{document}

\title{A user interaction based system browser}
\author{Roberto Minelli\\Benjamin Van Ryseghem}
%\date{}
\maketitle

\section{Introduction}

A system browser is the main tool of a Smalltalk developer as it provides easy access to the whole system source code.
From the early days of Smalltalk to today the system browser has proposed a workflow statically described by the system browser architects.
Due to this static description, the way to use the system browser is not really adapted to the user needs. 
Even if some implementation of this tool (OmniBrowser or Nautilus by example) proposed some mechanisms to bend the system browser workflow to fit the user workflow.
But so far this implementations were based on some extensions that the user, who is also a coder, can add by implementing new behaviours.

We would like to reimplement Nautilus (the current Pharo system browser) using Spec.
This new framework provides the possibility to dynamically change the user interface.
Such a feature combined with the ability to collect user interaction data provided by D-Flow will be used to make the system browser 
as close as the user interactions as possible.

The goal is to simplify the usage of the system browser which is the back bone tool of the system. 
To achieve this goal we would like to have a new system browser that can dynamically change the way information are presented in order to ease its usage for each user.

\section{Challenges}


\end{document}

% Roberto Minelli: Hi Ben
% Michele is in favor of the thingy with the system browser
% please provide me with a description of the project
% i.e., something that explains in details what will the project be about
% like, starting from the ground up..
% "The system browser lets the user...blah blah... we want to rewrite it using Spec because.. blah blah"
% "in the meantime we collect interaction data with which we want to enhance the new browser by means of blah blah