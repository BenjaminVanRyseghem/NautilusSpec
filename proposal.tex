\documentclass[12pt]{article}
\begin{document}

\title{An User Interaction Aware System Browser}
\author{Roberto Minelli\\Benjamin Van Ryseghem}
%\date{}
\maketitle

\section{Introduction}

A system browser is the main tool of a Smalltalk developer as it provides easy access to the whole system source code.
From the early days of Smalltalk to today the system browser has proposed a workflow statically described by the system browser architects.
Due to this static description, the way to use the system browser is not really adapted to the user needs. 
Even if some implementation of this tool (OmniBrowser or Nautilus by example) proposed some mechanisms to bend the system browser workflow to fit the user workflow.
But so far this implementations were based on some extensions that the user, who is also a coder, can add by implementing new behaviours.

We would like to reimplement Nautilus (the current Pharo system browser) using Spec.
This new framework provides the possibility to dynamically change the user interface.
Such a feature combined with the ability to collect user interaction data provided by D-Flow will be used to make the system browser 
as close as the user interactions as possible.

The goal is to simplify the usage of the system browser which is the back bone tool of the system. 
To achieve this goal we would like to have a new system browser that can dynamically change the way information are presented in order to ease its usage for each user.


\section{The current system browser}

The current system browser looks the same since the early ages of Smalltalk even if multiple implementation has existed.
It is horizontally split in two part:
\begin{itemize}
	\item the top most panel to navigate through the system;
	\item the bottom most panel where the source code is displayed and can be edited.
\end{itemize}

The top most panel is itself split in four columns representing the levels of Smalltalk entities.
From the left hand colin to the right hand column there are:
\begin{itemize}
	\item the packages list, including all the packages of the system
	\item the classes list of the selected package
	\item the protocols list of the current selected class
	\item the methods list of the current selected protocol
\end{itemize}


\section{Challenges}

Several challenges need to be faced in order to reach our goals.

The first challenge will be to filter the user interactions to a representative sample group. 
Among all the interactions made by the user only few are made often enough to be relevant.
The interactions need to be grouped by categories, allowing small variations within interactions of a same category.
Then only the relevant categories with a big enough population will be activated for interacting with the system browser workflow.

The second challenge will be to update the system browser workflow depending of the chosen categories.
To achieve this, an action must be defined for each category.
But in addition the actions must be able to be combined, or at least know how to be combine with another action.
Also some actions could occur only if multiple interactions categories are activated.

The last challenge will be to validate the actions.
Indeed it may be difficult to find a pertinent action for every user.
Then only a statistic approach mixed with tests on a sample group can lead to a generic enough solution.



\end{document}

% Roberto Minelli: Hi Ben
% Michele is in favor of the thingy with the system browser
% please provide me with a description of the project
% i.e., something that explains in details what will the project be about
% like, starting from the ground up..
% "The system browser lets the user...blah blah... we want to rewrite it using Spec because.. blah blah"
% "in the meantime we collect interaction data with which we want to enhance the new browser by means of blah blah