\include{amsymb}
\include{marvosym}

\usepackage[cyr]{aeguill}
\usepackage[applemac]{inputenc}
\usepackage{graphicx}
\usepackage{xspace}
\usepackage[a4paper]{geometry}
\usepackage{latexsym,amsmath,amssymb,textcomp}
\usepackage{moreverb}
\usepackage{listings}
\usepackage{multirow}
\usepackage{titling}
\usepackage{mathabx}

\usepackage{pdfpages}

\usepackage{hyperref}


\newlength{\indentationnota}
\newlength{\largeurlignenota}
\newlength{\paddingnota}
\newlength{\largeurnota}
\setlength{\paddingnota}{5pt}
\setlength{\largeurnota}{0.9cm}

\definecolor{pink}{rgb}{1,0.5,1}
\makeatletter

\newcommand{\figref}[1]{Figure~\ref{fig:#1}}
\newcommand{\figlabel}[1]{\label{fig:#1}}

\newenvironment{pictonote}[1]{% on passe le nom du fichier en argument
\begin{list}{}{%
\setlength{\labelsep}{0pt}%
\setlength{\rightmargin}{15pt}%
\setlength{\paddingnota}{5pt}}
\item%
\setlength{\indentationnota}{%
\@totalleftmargin+\largeurnota+\paddingnota}%
\setlength{\largeurlignenota}{%
\linewidth-\largeurnota-\paddingnota}%
\parshape=3%
\indentationnota\largeurlignenota%
\indentationnota\largeurlignenota%
\@totalleftmargin\linewidth%
\raisebox{-\largeurnota+2.2ex}[0pt][0pt]{%
\makebox[0pt][r]{%
\includegraphics[width=\largeurnota]{#1}%
\hspace{\paddingnota}}}%
\ignorespaces}{%
\end{list}}

\makeatother

\newenvironment{attention}%
{\begin{pictonote}{/Users/benjamin/Documents/Education/LaTeX/danger}}%
{\end{pictonote}}

\newenvironment{unicorn}%
{\begin{pictonote}{/Users/benjamin/Documents/Education/LaTeX/angry_unicorn}}%
{\end{pictonote}}

\newcommand\unicornbox[1]{\colorbox{pink}{\color{white}{#1}}}

\newcommand\cf{\emph{c.f}}

\newcommand\signature{%
\begin{figure}[br]
	\begin{flushright}
	\begin{minipage}{8cm}
		\begin{center}
		\reflectbox{\includegraphics[width=8cm]{/Users/benjamin/Documents/Education/LaTeX/unicorn}}\\
		\theauthor
		\end{center}
	\end{minipage}
	\end{flushright}
\end{figure}}

\newcommand\danger{\raisebox{-0.4ex}{\LARGE $\triangle$ \normalsize} \hspace{-4.1ex}! \hspace{1ex}}
\newcommand\PL{Programmation Logique\xspace}
\newcommand\pl{\bsc{Prolog}\xspace}
